We now describe local and asymptotic expansions of the Bessel function
$J_\nu(\omega r)$, and provide error analysis by which one can select the number
of terms needed in each expansion to assure $\epsilon$ accuracy in both regimes.

\subsection{The Wimp expansion}\label{sec:local}

Near the origin, $J_\nu(z)$ is a smooth and essentially non-oscillatory function
of $z$. As a result, $J_\nu(xy)$ is a numerically low-rank function of all
sufficiently small inputs $x$ and $y$. Fortuitously, one such low-rank expansion
--- which we refer to as the \textit{Wimp expansion} --- is available in closed
form for integer~$\nu$~\cite{wimp1962polynomial}. In the case that $\nu$ is
even, we have
\begin{equation}
    \begin{aligned}
        J_\nu(xy) 
        &= \sum_{\ell=0}^\infty C_\ell(x) \, T_{2\ell}(y) \\
        C_\ell(x) 
        &= \delta_\ell \, J_{\frac{\nu}{2} + \ell}(x) \, J_{\frac{\nu}{2} - \ell}(x) \\
        \delta_\ell 
        &= \begin{cases}
            1 & \ell=0 \\
            2 & \text{otherwise}
        \end{cases}
    \end{aligned}
\end{equation}
for all $\abs{y} \leq 1$. A similar expansion exists for $\nu$
odd~\cite[2.23]{wimp1962polynomial}.

In order to employ the Wimp expansion to compute local terms within the Hankel
transform, we must determine the number of terms $L$ needed to construct an
$\epsilon$-accurate approximation to $J_\nu(\omega r)$ on a given rectangle
$(\omega, r) \in [0, \Omega] \times [0, R]$. The following lemma provides a
bound on the induced truncation error in the Wimp expansion as a function of the
order $\nu$, the space-frequency product $\Omega R$, and the number of retained
terms $L$.

\begin{lemma} \label{lem:wimp} Truncating the Wimp expansion after $L$ terms
    gives
    \begin{align} \label{eq:loc-error}
        \abs{J_\nu(\omega r) - \sum_{\ell=0}^L C_\ell(\omega R) T_{2\ell}\left( \frac{r}{R} \right)} 
        \leq \frac{2\exp\left\{ \frac{\nu}{2}(\beta - \gamma) + (L+1)(\beta + \gamma) \right\}}{1 - e^{\beta + \gamma}}
        =: \Bloc(\Omega R)
    \end{align}
    for all $\omega \in [0, \Omega], r \in [0, R]$, where
    \begin{align}
        \psi(p) &:= \log p + \sqrt{1 - p^2} - \log\left( 1 + \sqrt{1 - p^2} \right) \\
        \beta &:= \psi\left( \frac{\Omega R}{2L + 2 + \nu} \right) \\
        \gamma &:= \begin{cases}
            \psi\left( \frac{\Omega R}{2L + 2 - \nu} \right) & L + 1 \geq \frac{\nu}{2} \\
            0 & \textnormal{otherwise}
        \end{cases}
    \end{align}
\end{lemma}
\begin{proof}
    For $\nu$ even, the truncation error after $L$ terms is bounded by
    \begin{align}
        \abs{\sum_{\ell=L+1}^\infty C_\ell(\omega R) T_{2\ell}\left( \frac{r}{R} \right)}
        &\leq 2\sum_{\ell=L+1}^\infty \abs{J_{\frac{\nu}{2} + \ell}\left(\frac{\omega R}{2}\right)} \abs{J_{\frac{\nu}{2} - \ell}\left(\frac{\omega R}{2}\right)}.
    \end{align}
    Define $p_\ell(\omega) := \omega R / (\nu + 2\ell)$. Then by Siegel's bound
    \cite[10.14.5]{olver2010nist} we have
    \begin{align}
        \abs{J_{\frac{\nu}{2} + \ell}\left(\frac{\omega R}{2}\right)}
        &= \abs{J_{\frac{\nu}{2} + \ell}\bigg(\Big(\frac{\nu}{2} + \ell\Big) p_\ell(\omega)\bigg)} \\
        &\leq \exp\left\{ \Big(\frac{\nu}{2} + \ell\Big) \psi\big(p_\ell(\omega)\big)\right\} \\
        &\leq \exp\left\{ \Big(\frac{\nu}{2} + \ell\Big) \beta \right\},
    \end{align}
    where the last inequality follows from the fact that $\psi$ is an increasing
    function on $(0,1)$, and thus $\psi\big(p_\ell(\omega)\big) \leq \beta < 0$
    for all $\ell \geq L+1$ and all $\omega \in [0, \Omega]$. 
    
    If $L+1 \geq \frac{\nu}{2}$, we define $q_\ell(\omega) := \omega R / (2\ell
    - \nu)$ and apply Siegel's bound again to obtain
    \begin{align}
        \abs{J_{\frac{\nu}{2} - \ell}\left(\frac{\omega R}{2}\right)}
        &= \abs{J_{\ell - \frac{\nu}{2}}\bigg(\Big(\ell - \frac{\nu}{2}\Big) q_\ell(\omega)\bigg)} 
        \leq \exp\left\{ \Big(\ell - \frac{\nu}{2}\Big) \gamma \right\}.
    \end{align}
    If $L+1 < \frac{\nu}{2}$, Siegel's bound does not apply and we use instead
    the simple bound $\abs{J_{\frac{\nu}{2} - \ell}\left(\frac{\omega
    R}{2}\right)} \leq 1$, which is equivalent to taking $\gamma = 0$. 

    All that remains is to apply a geometric series argument
    \begin{align}
        \abs{J_\nu(\omega r) - \sum_{\ell=0}^L C_\ell(\omega R) T_{2\ell}\left( \frac{r}{R} \right)}
        &\leq 2\sum_{\ell=L+1}^\infty \exp\left\{ \Big(\frac{\nu}{2} + \ell\Big) \beta + \Big(\ell - \frac{\nu}{2}\Big) \gamma \right\} \\
        &= 2\exp\left\{\frac{\nu}{2}(\beta - \gamma)\right\} \sum_{\ell=L+1}^\infty \left( e^{\beta + \gamma} \right)^\ell \\
        &= \frac{2\exp\left\{ \frac{\nu}{2}(\beta - \gamma) + (L+1)(\beta + \gamma) \right\}}{1 - e^{\beta + \gamma}}
    \end{align}
    A similar calculation can be carried out for $\nu$ odd.
\end{proof}

Lemma~\ref{lem:wimp} is rather opaque regarding the impact of the various
parameters on the error because we have not utilized any simplifying bounds on
the function $\psi$, as done in~\cite[Lemma 1]{rangan2020factorization} for
large $\nu$. However, our analysis takes into account the decay in both
$J_{\frac{\nu}{2}+\ell}$ \textit{and} $J_{\frac{\nu}{2}-\ell}$, thus remaining
relatively tight for small $\nu$. It is therefore well-suited to our purposes
because, given $z, L > 0$, it provides a bound $\Bloc(z)$ on the pointwise error
in approximating any block of the matrix $J_\nu(\omega_j r_k)$ for which $\omega
r \leq z$ using the $L$-term Wimp expansion. 

This expansion is highly beneficial from a computational perspective, as it
yields an analytical rank-$L$ approximation to any block of $\bm{A}$ for which
$\omega_j r_k$ is sufficiently small
\begin{equation}
  \mtx{A}(j_0:j_1, k_0:k_1)
  %\vct{f}(k_0:k_1)
  \approx \mtx{C}\mtx{T}^\top
  %\vct{f}(k_0:k_1)
\end{equation}
where $\mtx{C} \in \R^{(j_1-j_0+1) \times L}$ and $\mtx{T} \in \R^{(k_1-k_0+1)
  \times L}$ with entries
\begin{equation}
  \mtx{C}(j,\ell) = C_{\ell-1}(\omega_j r_{k_1}) \qquad  \text{and} \qquad 
  \mtx{T}(k,\ell) = T_{2\ell-2}\left(\frac{r_k}{r_{k_1}}\right).
\end{equation}
For a block of $\mtx{A}$ of size $m_b \times n_b$, the low-rank approximation
given by the Wimp expansion can be applied to a vector in $\bO\big(L(m_b +
n_b)\big)$ time by first applying $\bm{T}^\top$ then applying $\bm{C}$.

\subsection{Hankel's expansion}
\label{sec:asymptotic}

Away from the origin, $J_\nu(z)$ exhibits essentially sinusoidal oscillation
with period $2\pi$. This statement is made precise by Hankel's asymptotic
expansion, which states that for $z \to \infty$
\begin{align} \label{eq:asymptotic-expansion}
    J_\nu(x)
    \sim \sqrt{\frac{2}{\pi x}} \left( 
        \cos\left(x + \phi\right) \sum_{\ell=0}^{\infty} \frac{(-1)^\ell a_{2\ell}(\nu)}{x^{2\ell}}
        - \sin\left(x + \phi\right) \sum_{\ell=0}^{\infty} \frac{(-1)^\ell a_{2\ell+1}(\nu)}{x^{2\ell+1}}
        \right)
\end{align}
where $\phi := - \frac{(2\nu+1)\pi}{4}$ and 
\begin{align}
    a_\ell(\nu) := \frac{(4\nu^2 - 1)(4\nu^2 - 3)\dots(4\nu^2 -
  (2\ell-1)^2)}{\ell! \, 8^\ell}.
\end{align}
Rearranging this expansion, we obtain an expansion which can be evaluated using
two NUFFTs and diagonal scalings, and whose remainder is bounded by the size of
the first neglected terms~\cite[Section 7.3]{watson1922treatise}
\begin{multline} \label{eq:asy-error}
    \Bigg| J_\nu(\omega r)
    - \sqrt{\frac{2}{\pi}} \sum_{\ell=0}^{M-1} \left[  
        \frac{(-1)^\ell a_{2\ell}(\nu)}{\omega^{2\ell + \frac{1}{2}}} \Re\left(
          \frac{e^{i(\omega r + \phi)}}{r^{2\ell+\frac{1}{2}}}\right)  - \frac{(-1)^\ell
            a_{2\ell+1}(\nu)}{\omega^{2\ell+\frac{3}{2}}}
          \Im\left(\frac{e^{i(\omega r + \phi)}}{r^{2\ell+\frac{3}{2}}} \right) \right]
        \Bigg| \\
         \leq \sqrt{\frac{2}{\pi}} \left( \frac{\abs{a_{2M}(\nu)}}{(\omega r)^{2M+\frac{1}{2}}} + \frac{\abs{a_{2M+1}(\nu)}}{(\omega r)^{2M+\frac{3}{2}}} \right) =: \Basy(z)
\end{multline}

The computational advantage of this expansion is that the $2M$-term asymptotic
expansion of any block of $\bm{A}$ can be rapidly applied to a vector~$\vct{x}$
using $2M$ Type-III NUFFTs
\begin{multline}
    \mtx{A}(j_0:j_1, k_0:k_1) \vct{x} 
    \approx \sqrt{\frac{2}{\pi}} \sum_{\ell=0}^{M-1} (-1)^\ell \Bigg[ 
        a_{2\ell}(\nu) \mtx{D}_\omega^{-2\ell-\frac{1}{2}} \Re\Big(e^{i\phi}
                                                             \mtx{F}
                                                             \mtx{D}_r^{-2\ell-\frac{1}{2}}
                                                             \vct{x} \Big) \\
        - a_{2\ell+1}(\nu) \mtx{D}_\omega^{-2\ell-\frac{3}{2}}
                                                             \Im\Big(e^{i\phi}
                                                               \mtx{F}
                                                               \mtx{D}_r^{-2\ell-\frac{3}{2}}
                                                               \vct{x} \Big) 
    \Bigg]
\end{multline}
where $\mtx{F} \in \C^{(j_1 - j_0 + 1) \times (k_1 - k_0 + 1)}$ is the Type-III
nonuniform DFT matrix corresponding to frequencies $\omega_{j_0}, \dots,
\omega_{j_1}$ and points $r_{k_0}, \dots, r_{k_1}$, and the diagonal scaling
matrices are given by $\mtx{D}_\omega :=
\diag(\omega_{j_0},\dots,\omega_{j_1})$, and $\mtx{D}_r :=
\diag(r_{k_0},\dots,r_{k_1})$.

\subsection{Determining order of expansions and crossover point}
\label{sec:cutoff}

With these error bounds in hand, we precompute the parameters~$z_{\nu,
\epsilon}^M$ and~$L_{\nu, \epsilon}^M$ for tolerances $\epsilon = 10^{-4},
\dots, \allowbreak 10^{-15}$, orders $\nu = 1, \dots, 100$, and number of
asymptotic expansion terms $M = 1, \dots, 20$:
\begin{itemize}
    \item $z_{\nu, \epsilon}^M$ such that $M$-term Hankel expansion of
    $J_\nu(\omega r)$ is $\epsilon$-accurate $\forall \ \omega r > z_{\nu,
    \epsilon}^M$,
    \item $L_{\nu, \epsilon}^M$ such that $L_{\nu, \epsilon}^M$-term Wimp
    expansion of $J_\nu(\omega r)$ is $\epsilon$-accurate $\forall \ \omega r
    \leq z_{\nu, \epsilon}^M$.
\end{itemize}
First, the crossover points $z_{\nu, \epsilon}^M$ are computed using Newton's
method on the function $\xi(z) := \Basy(z) - \epsilon$. Then the number of local
expansion terms $L_{\nu, \epsilon}^M$ are taken to be the smallest integer such
that $\Bloc\left(z_{\nu, \epsilon}^M\right) < \epsilon$. These tables are
precomputed once when the library is installed, and even this precomputation
requires only a few seconds on a laptop.

With these tables stored, for any order $\nu$ we can look up a pair of
complementary local and asymptotic expansions with error everywhere bounded by
the requested tolerance $\epsilon$. The only remaining free parameter is the
number of asymptotic terms $M$. This parameter is selected based on various
numerical experiments which maximize speed by balancing the cost of the local,
asymptotic, and direct evaluations. In our implementation, we use the heuristic
\begin{equation}
  \label{eq:num-asy-terms}
    M = \min\left(\flr{1 + \frac{\nu}{5} - \frac{\log_{10}(\epsilon)}{4}}, 20\right).
\end{equation}




%%% Local Variables: % mode: latex % TeX-master: "../main" % End:
