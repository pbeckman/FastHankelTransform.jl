\subsection{Comparison to direct evaluation}

\red{
  Scaling and accuracy tests against direct evaluation using point distributions for which $P = \bO(n)$
  \begin{itemize}
    \item Sch\"olmilch
    \item Fourier-Bessel
    \item Scaled roots (DHT)
  \end{itemize}
  
  for which $n$ grows with $m$ and $P$ constant (more sources in fixed domain)
  
  for which $m$ grows with $n$ and $P$ constant (more targets in fixed domain)
  
  for which $P$ grows with $n$ and $m$ constant (spreading out points)
}

\subsection{Computing Fourier transforms of radial functions}

For radial functions $f(\bm{r}) = f(\norm{\bm{r}})$ in $\R^d$, one can integrate
out the radial variables analytically, reducing the $d$-dimensional Fourier
integral to a single Hankel transform
\begin{align} \label{eq:radial-fourier}
    \hat{f}(\bm{\omega}) 
    = \int_{\R^d} f(\norm{\bm{r}}) e^{i\bm{\omega}^\top \bm{r}} \dif{\bm{r}}
    = \frac{(2\pi)^{\frac{d}{2}}}{\omega^{\frac{d}{2} - 1}} \int_0^\infty f(r) J_{\frac{d}{2} - 1}(\omega r) r^{\frac{d}{2}} \dif{r}.
\end{align}

\subsubsection{Two dimensions}

We compare two methods of computing $\hat{f}$ for the indicator function of the
unit disk $f(r) = \ind{0 \leq r \leq 1}$ to absolute error $\epsilon = 10^{-12}$
at $n$ equispaced points $\omega_j \in [0, \omega_{\text{max}}]$. First, we use
a Gauss-Legendre quadrature rule on $[0,1]$ with nodes $r_k$ and weights $w_k$.
We utilize the NUFHT to compute the resulting sum
\begin{align}
  \hat{f}(\omega) 
  &= 2\pi\int_0^1 f(r) J_0(\omega r) r \dif{r} \\
  &\approx 2\pi \sum_{k=1}^m w_k f(r_k) J_0(\omega r_k) r_k,
\end{align}
doubling the number of nodes $m$ until the error in the computed integral is
less than $\epsilon$. Second, we use a tensor product quadrature rule in polar
coordinates, using the same $m$-point Gauss-Legendre rule in $r$ and a $t$-node
trapezoidal rule in $\theta$. We utilize the NUFFT to compute the resulting
double sum
\begin{align}
  \hat{f}(\omega) 
  &= \int_0^{2\pi} \int_0^1 f(r) e^{i\omega r\cos\theta} r \dif{r} \dif{\theta} \\
  &\approx \frac{2\pi}{t} \sum_{k=1}^{m} \sum_{s=1}^{t} w_k f(r_k) \exp\left\{i\omega r_k\cos\left(\frac{2\pi s}{t}\right)\right\} r_k,
\end{align}
doubling the number of trapezoidal nodes $t$ until the error in the computed
integral is less than $\epsilon$.

If only low frequencies $\omega$ are desired, e.g. $\omega_{\text{max}} = 100$,
the integrands are only mildly oscillatory and few trapezoidal nodes are
required. In combination with the relative ease of amortizing costs in the
NUFFT, the two-dimensional transform is often faster than the NUFHT. However,
for larger $\omega_{\text{max}}$ the integrands become more oscillatory, and
more nodes are needed in each dimension in order to resolve them. In such cases
the fact that $\bO(m^2)$ quadrature nodes are needed in $\R^2$ becomes a
bottleneck, and the necessary NUFFT becomes prohibitively large. 

\red{Quadrature and timing figures}

\subsubsection{Higher dimenions}

Using spherical Bessel function identities \cite[10.47.3,
10.49.2]{olver2010nist}, one can see that for half-integer $\nu$, Hankel's
expansion (\ref{eq:asymptotic-expansion}) is no longer just an asymptotic
expansion, but an exact formula.
% \begin{align} J_{\frac{d}{2}-1}(z) &= \sqrt{\frac{2z}{\pi}}
%   j_{\frac{d-3}{2}}(z) \\
%   &= \sqrt{\frac{2}{\pi z}} \left( \cos\left(\mu\right) \sum_{\ell=0}^{M-1}
%     (-1)^\ell \frac{a_{2\ell}\left(\frac{d}{2}-1\right)}{z^{2\ell}} -
%     \sin\left(\mu\right) \sum_{\ell=0}^{M-1} (-1)^\ell
%     \frac{a_{2\ell+1}\left(\frac{d}{2}-1\right)}{z^{2\ell+1}} \right)
%     \end{align} where $\mu := z - \frac{(2\nu+1)\pi}{4}$. 
For example, the kernel of the integral transform in (\ref{eq:radial-fourier})
for $d = 3$ is
\begin{align}
  J_{\frac{1}{2}}(z) = \sqrt{\frac{2}{\pi z}} \sin(z),
\end{align}
and the corresponding NUFHT can be evaluated using a single NUFFT.

The curse of dimensionality demonstrated in the two-dimensional case only
intensifies in higher dimensions, for which the requirement of $\bO(m^d)$
quadrature nodes is infeasible for even small $m$. In contrast, the NUFHT only
ever requires one-dimensional quadrature, and is relatively robust to the
distribution of $\omega_j$ and $r_k$, and to dimension for $d \leq 20$ or so. In
even higher dimensions $d > 20$, the number of terms in the Hankel and Wimp
expansions needed to maintain accurate evaluation increases, and the methods
presented here become inefficient in practice, despite retaining their
quasilinear asymptotic complexity. There exist several alternative asymptotic
expansions in this large $\nu$ regime which can be leveraged for fast evaluation
of $J_\nu(z)$~\cite{heitman2015asymptotics, olver2010nist}.
%\cite[10.19]{heitman2015asymptotics, olver2010nist}
However, the terms in these asymptotics are not sinusoids which can be
efficiently evaluated with the NUFFT, and thus turning them into an
analysis-based fast transform remains an open problem.
