
The fast Fourier transform (FFT) has revolutionized a wide range of applications
across mathematics, statistics, and the physical sciences by enabling signal
processing and Fourier analysis tasks to be performed using a computational cost
which scales quasi-linearly with the number of data points~$n$. However, the FFT
requires that the input signal be sampled at equispaced points in time and that
the desired output frequencies are equispaced on the integers. These assumptions
are frequently not met in applications such as adaptive numerical partial
differential equation (PDE)
solvers~\cite{alpert2002adaptive,jiang2023dual,askham2017adaptive,nochetto2009theory},
magnetic resonance
imaging~\cite{greengard2007fast,bondesson2019nonuniform,bronstein2002reconstruction},
and various signal processing
tasks~\cite{alexander2012adaptive,thakur2011synchrosqueezing}. To overcome this
setback, nonuniform FFT (NUFFT) algorithms have been
developed~\cite{dutt1993fast,greengard2004accelerating} which achieve near-FFT
speeds in one dimension, assuming that the distribution of time samples and
frequency outputs is not pathological. In higher dimensions, NUFFTs are less
competitive with standard FFTs, but the computational task at hand is also
significantly harder.

The FFT and NUFFT grew out of a need to perform Fourier transforms in Cartesian
coordinates. However, depending on the particular problem, the relevant
continuous Fourier analysis might be better suited to other coordinate systems.
One such commonly encountered situation is computing the Fourier transform of
radially symmetric functions in dimensions~$d \geq 2$. For example, in two
dimensions the Fourier transform of a function~$f$ is given by
\begin{equation}
  g(\omega_1, \omega_2) = \frac{1}{4\pi^2} \iint_{\R^2} f(x_1, x_2) \, 
  e^{-i(\omega_1 x_1 + \omega_2 x_2)}  \, dx_1 \, dx_2.
\end{equation}
Transforming to polar coordinates~$(\omega_1,\omega_2) \mapsto (\omega,\alpha)$
and~$(x_1,x_2) \mapsto (r,\theta)$ the above expression becomes
\begin{equation}
  \label{eq:ftpolar}
  \begin{aligned}
    g(\omega, \alpha) &= \frac{1}{4\pi^2} \int_0^{2\pi} \int_0^\infty
    f(r,\theta) \, 
    e^{-i \omega r (\cos\alpha \cos\theta + \sin\alpha \sin\theta) } 
    \, r \, dr \, d\theta \\
  &= \frac{1}{4\pi^2} \int_0^{2\pi} \int_0^\infty f(r,\theta) \, e^{-i \omega r \cos(\alpha-\theta) } \, r \, dr \, d\theta.
  \end{aligned}
\end{equation}
Furthermore, if~$f$ is radially symmetric, i.e.~$f(r, \theta) = f(r)$, then the
above transform can be written as
\begin{equation}
  \label{eq:HT}
  \begin{aligned}
  g(\omega,\alpha) &= \frac{1}{4\pi^2} \int_0^\infty f(r) \, r \int_0^{2\pi} 
  e^{-i \omega r \cos(\alpha - \theta) }  \, d\theta \, dr \\
  &= \frac{1}{2\pi} \int_0^\infty f(r) \, J_0(\omega r) \, r \, dr,
  \end{aligned}
\end{equation}
where we have used the integral representation of the zeroth-order Bessel
function~\cite{olver2010nist}
\begin{equation}
  J_0(x) 
  = \frac{1}{\pi} \int_0^\pi \cos \left( x \cos \theta \right) \, d\theta.
\end{equation}
The final integral involving~$J_0$ in equation~\eqref{eq:HT} is known as a
\emph{Hankel Transform} of order 0 --- usually referred to simply as a Hankel
Transform. 

In higher ambient dimensions, the Fourier transform of radially symmetric
functions reduces to a Hankel transform of higher order. Similarly, if the
function~$f$ in~\eqref{eq:ftpolar} has a particular periodic dependence
in~$\theta$ so that $f(r,\theta) = f(r)e^{i\nu\theta}$ with $\nu \in \Z$, then
we have
\begin{equation} \label{eq:FB-integral}
  \begin{aligned}
  g(\omega,\alpha) &= \frac{1}{4\pi^2} \int_0^\infty f(r) \, r \int_0^{2\pi} 
  e^{-i \omega r \cos(\alpha - \theta) } \, e^{i\nu\theta}  \, d\theta \, dr \\
  &= \frac{i^\nu}{2\pi} \int_0^\infty f(r) \, r \, J_\nu(\omega r)  \, dr,
  \end{aligned}
\end{equation}
where, again, we have invoked an integral representation
for~$J_\nu$~\cite{olver2010nist}. 

In order to numerically compute~$g$ in~\eqref{eq:HT} or~\eqref{eq:FB-integral}
at a collection of~$m$ ``frequencies''~$\omega_j$, the Hankel transform must be
discretized using an appropriate quadrature rule with nodes $r_k$ and weights
$w_k$ which depend on the particular class of~$f$ for which the integral is
desired. In general this results in the need for computing
\begin{equation} \label{eq:DHT}
  \begin{aligned}
  g(\omega_j) \approx 
  g_j &:= \sum_{k=1}^n w_k \, f(r_k) \, r_k \, J_\nu(\omega_j r_k) \\
  &\ = \sum_{k=1}^n c_k \, J_\nu(\omega_j r_k)
   \qquad \text{for } j = 1, \ldots, m.
  \end{aligned}
\end{equation}
The above sum will be referred to as the Discrete Hankel Transform (DHT) of
order $\nu$. 

In our motivating example --- computing the continuous Fourier transform --- the
DHT arises from the discretization of the radially symmetric Fourier integral.
The DHT also appears in a wide range of applications including
imaging~\cite{higgins1988hankel, zhao2013fourier, marshall2023fast},
statistics~\cite{lord1954a, genton2002nonparametric}, and separation of
variables methods in partial differential
equations~\cite{bisseling1985fast,ali1999generalized, zhou2022spectral}. In many
such applications, a fully nonuniform DHT is desired, as the relevant
frequencies $\omega_j$ may not be equispaced, and the most efficient quadrature
rule for discretizing (\ref{eq:HT}) may have nodes $r_k$ which are also not
equispaced. 

The algorithm of this work allows for arbitrary selection of the
frequencies~$\omega_j$ and nodes~$r_k$, in contrast to other algorithms which
require some structure to their location (e.g. equispaced or exponentially
distributed). There are a few types of commonly encountered DHTs, all of which
our algorithm can address. Schl\"omilch
expansions~\cite{linton2006schlomilch,townsend2015fast} take
frequencies~$\omega_j = j\pi$. Fourier-Bessel expansions --- often used in
separation of variables calculations for PDEs --- take frequencies~$\omega_j =
\beta_{\nu,j}$, where $\beta_{\nu,j}$ denotes the $j^{th}$ root of $J_\nu$. In
the most restrictive cases~\cite{johnson1987improved}, one fixes both~$\omega_j
= \beta_{\nu,j}$ and~$r_k = \beta_{\nu,k}/\beta_{\nu,k+1}$.

\subsection*{Existing methods}
\label{sec:existing}

A number of methods exist in the literature to evaluate (\ref{eq:HT}) and
(\ref{eq:DHT}). These include series expansion methods
\cite{lord1954b,brunol1977fourier,cavanagh1979numerical}, convolutional
approaches \cite{siegman1977quasi, johansen1979fast, mook1983algorithm,
liu1999nonuniform}, and projection-slice or Abel transform-based methods
\cite{oppenheim1980computation, hansen1985fast, kapur1995algorithm}.
See~\cite{cree1993algorithms} for a review of many of these early computational
approaches. Unfortunately, these existing methods are either not applicable to
the discrete case, require a particular choice of $\omega_j$ or $r_k$ due to the
constraints of interpolation or quadrature subroutines, or suffer from low
accuracy as a result of intermediate approximations. Therefore, extending these
schemes to compute the fully nonuniform DHT with controllable accuracy is not
straightforward.

A notable contribution is~\cite{liu1999nonuniform}, which describes a fully
nonuniform fast Hankel transform. This work takes the popular convolutional
approach, using a change of variables to reformulate the Hankel transform as a
convolution with a known kernel which can be evaluated using the NUFFT. However,
its accuracy is limited by the need for a quadrature rule on the nonuniform
points $r_k$. The authors use an irregular trapezoidal rule for this purpose,
which is not high-order accurate. This method also requires the computation of
the inverse NUFFT using conjugate gradients. For even moderately clustered
points or frequencies, this inverse problem is extremely ill-conditioned, and
thus the number of required iterations can be prohibitive. This method is
therefore suitable for ``quasi-equispaced'' points and frequencies, but is not
tractable in general.

More recently, butterfly algorithms \cite{oneil2010algorithm, li2015butterfly,
pang2020interpolative} were introduced as a broadly applicable methodology for
rapidly computing oscillatory transforms including the nonuniform DHT. However,
these algorithms require a precomputation or factorization stage for each new
set of $\omega_j$ and $r_k$. Such precomputations can, unfortunately, be a
bottleneck for applications in which these evaluation points change with each
iteration or application of the transform. In order to provide a
precomputation-free fast DHT,~\cite{townsend2015fast} employs a combination of
asymptotic expansions and Bessel function identities evaluated using the
equispaced FFT. The resulting scheme is applicable to equispaced or perturbed
``quasi-equispaced'' grids in space and frequency, for example $\omega_j =
\beta_{0,j}$ and $r_k = \beta_{0,k} / \beta_{0,n+1}$.

\subsection*{Novelty of this work}
\label{sec:novelty}

% At a high level, our algorithm can be viewed as a generalization of the one
% described in~\cite{townsend2015fast}. In~\cite{townsend2015fast}, asymptotic
% expansions were used to replace~$J_0$ for various arguments. These asymptotic
% expansions involved trigonometic functions, resulting in a fast algorithm for
% computing the DHT using fast cosine transforms (FCTs) and fast sine transforms
% (FSTs). In order to invoke these fast algorithms, various assumptions
% on~$\omega_j$ and~$r_k$ had to be made.

We describe here a precomputation-free nonuniform fast Hankel transform (NUFHT)
which generalizes~\cite{townsend2015fast} to the fully nonuniform setting in a
number of ways. First, we employ an adaptive partitioning scheme which, for any
choice of $\omega_j$ and $r_k$, subdivides the matrix with
entries~$J_\nu(\omega_j r_k)$ into blocks for which matrix-vector products can
be evaluated efficiently. Second, we use the NUFFT to evaluate asymptotic
expansions for nonuniform $r_k$ and $\omega_j$. Finally, we utilize the low-rank
expansion of $J_\nu$ given in~\cite{wimp1962polynomial} in the local regime
where asymptotic expansions are not applicable. We derive error bounds for this
low-rank expansion, allowing us to choose all approximation parameters
automatically by analysis which guarantees that the resulting error is bounded
by the user-specified tolerance $\epsilon$.

\subsection*{Outline of the paper}

The paper is organized as follows. In Section~\ref{sec:overview} we give a high
level view of our algorithm, omitting technical details. Then in
Section~\ref{sec:approx} we study the local and asymptotic expansions of Bessel
functions which serve as the key building blocks of the algorithm. Afterward, in
Section~\ref{sec:methods}, we provide a detailed description of the algorithm
and its associated complexity. Various numerical examples are provided in
Section~\ref{sec:results}, and we conclude with some additional discussion in
Section~\ref{sec:discussion}.




%%% Local Variables: %% mode: latex %% TeX-master: "../main" %% End:
