The Hankel or Fourier-Bessel transform of order $\nu$ of a function $f : (0,
\infty) \to \C$ is given by
\begin{equation} \label{eq:HT}
    g(\omega) = \int_0^\infty f(r) J_\nu(\omega r) r \dif{r}.
\end{equation}
Discretizing this integral, we obtain the discrete Hankel transform (DHT)
\begin{equation} \label{eq:DHT}
    g_j = \sum_{k=1}^m c_k J_\nu(\omega_j r_k)
\end{equation}
for $j=1, \dots, n$, which can be written equivalently as a matrix-vector
product with $\bm{A} \in \R^{n \times m}$
\begin{equation}
    \bm{g} = \bm{A}\bm{c}, \qquad \bm{A}_{jk} = J_\nu(\omega_j r_k).
\end{equation}
These transforms appear in a wide range of applications including
imaging~\citep{higgins1988hankel, zhao2013fourier}, partial differential
equations~\citep{bisseling1985fast,ali1999generalized}, and
statistics~\cite{lord1954a, genton2002nonparametric}. In many such applications,
a fully nonuniform DHT is desired, as the relevant frequencies $\omega_j$ may
not be equispaced, and the most efficient quadrature rule for discretizing
(\ref{eq:HT}) may have nodes $r_k$ which are also not equispaced.

A number of methods exist in the literature to evaluate (\ref{eq:HT}) or
(\ref{eq:DHT}). These include series expansion methods
\citep{lord1954b,brunol1977fourier,cavanagh1979numerical}, convolutional
approaches \citep{siegman1977quasi, johansen1979fast, mook1983algorithm}, and
projection-slice or Abel transform-based methods
\citep{oppenheim1980computation, hansen1985fast, kapur1995algorithm}.
See~\cite{cree1993algorithms} for a review of many of these early computational
approaches. Unfortunately, these existing methods are either not applicable to
the discrete case (\ref{eq:DHT}), require a particular choice of $\omega_j$ or
$r_k$ due to the constraints of interpolation or quadrature subroutines, or
suffer from low accuracy as a result of intermediate approximations. Therefore,
extending these schemes to compute the fully nonuniform DHT with controllable
accuracy is not straightforward. 

More recently, butterfly algorithms \citep{oneil2010algorithm, li2015butterfly,
pang2020interpolative} were introduced as a broadly applicable methodology for
rapidly computing oscillatory transforms including the nonuniform DHT. However,
these algorithms require a precomputation or factorization stage for each new
set of $\omega_j$ and $r_k$, which can be a bottleneck for applications in which
these evaluation points change with each evaluation. In order to provide a
precomputation-free fast DHT,~\cite{townsend2015fast} uses a combination of
asymptotic expansions and Bessel function identities applicable to equispaced or
perturbed ``quasi-equispaced'' grids, for example $j_{\nu,k} / j_{\nu,n+1}$
where $j_{\nu,k}$ is the $k^{th}$ zero of $J_\nu$.

We describe here a fully nonuniform fast Hankel transform (NUFHT) which
generalizes~\cite{townsend2015fast} in a number of ways. First, we employ an
adaptive partitioning scheme which, for any choice of $r_k$ and $\omega_j$,
subdivides $\bm{A}$ into blocks for which matrix-vector products can be
evaluated efficiently. Second, we use the nonuniform fast Fourier transform
(NUFFT) \citep{dutt1993fast, greengard2004accelerating} to evaluate asymptotic
expansions for nonuniform $r_k$ and $\omega_j$. Finally, we utilize the low-rank
expansion of \cite{wimp1962polynomial} in the local regime where asymptotic
expansions are not applicable. We derive error bounds for this low-rank
expansion, allowing us to choose all approximation parameters automatically
through error analysis which guarantees that the resulting error is bounded by
the user-specified tolerance $\epsilon$. 

